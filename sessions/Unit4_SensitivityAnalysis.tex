%%%%%%%%%%%%%%%%%%%%%%%%%%%%%%%%%%%%%%%%%
% Beamer Presentation
% LaTeX Template
% Version 1.0 (10/11/12)
%
% This template has been downloaded from:
% http://www.LaTeXTemplates.com
%
% License:
% CC BY-NC-SA 3.0 (http://creativecommons.org/licenses/by-nc-sa/3.0/)
%
%%%%%%%%%%%%%%%%%%%%%%%%%%%%%%%%%%%%%%%%%

%----------------------------------------------------------------------------------------
%	PACKAGES AND THEMES
%----------------------------------------------------------------------------------------

\documentclass[c]{beamer}
%\documentclass[notes]{beamer}
\setbeamertemplate{note page}[show only notes]
\input{OR_common.tex}
%%%%%%%%%%%%%%%%%%%%%%%%%%%%%%%%%%%%%%%%%%%%%%%%%%%%%%%%%%%%%%%%%%%%%%%%%%%%%
%%%%%%%%%%%%%%%%%%%%%%%%%%%%%%%%%%%%%%%%%%%%%%%%%%%%%%%%%%%%%%%%%%%%%%%%%%%%%
%%%%%%%%%%%%%%%%%%%%%%%%%%%%%%%%%%%%%%%%%%%%%%%%%%%%%%%%%%%%%%%%%%%%%%%%%%%%%

\title[Introduction]{Unit 4. Sensitivity Analysis}

\author{Jordi Villà i Freixa}
\institute[FCTE]{
Universitat de Vic - Universitat Central de Catalunya \\
Study Abroad. Operations Research\\
\medskip
\textit{jordi.villa@uvic.cat\\\url{https://mon.uvic.cat/cbbl}}
}
\date{November 12th, 2025\\last updated: \today}
\logo{\includegraphics[width=.1\textwidth]{FCTE}}
\begin{document}

\begin{frame}
\titlepage
\end{frame}

\begin{frame}
    \frametitle{Preliminary}
    This course is strongly based on the monography on Operations Research by Carter, Price and Rabadi \cite{carter}, and in material obtained from different sources (quoted when needed through the slides).
\end{frame}

%%%%%%%%%%%%%%%%%%%%%%%%%%%%%%%%%%%%%%%%%%%%%%%%%%%%%%%%%%%%%%%%%%%%%%%%%%%%%
%%%%%%%%%%%%%%%%%%%%%%%%%%%%%%%%%%%%%%%%%%%%%%%%%%%%%%%%%%%%%%%%%%%%%%%%%%%%%
%%%%%%%%%%%%%%%%%%%%%%%%%%%%%%%%%%%%%%%%%%%%%%%%%%%%%%%%%%%%%%%%%%%%%%%%%%%%%

\begin{frame}
\frametitle{Learning outcomes}
\begin{itemize}
  \item Understanding the concept of postoptimality analysis
  \item Understanding and solving LP problems requiring sensitvity analysis
  \item Understanding the matrix representation of the Simplex solution
\end{itemize}
\end{frame}

\begin{frame}{The concept}
\begin{itemize}
  \item After an optimal solution is found, the analyst needs to review the problem parameters and the solution.
  \item This process is called {\bf postoptimality analysis}:
  \begin{itemize}
    \item confirming or updating problem parameters (cost and availabilities of activities and resources)
    \item if changes need to be introduced in the original parameters, assessing their impact on the optimality of the solution.
  \end{itemize}
  \item If changes are small, re-optimization may not be needed.
  \item {\bf Sensitivity analysis} is the study of the effect that types, ranges, and magnitude of changes in problem parameters have in the value of the objective function, {\em without the need to solve again the new linear problem}.
\end{itemize}
\end{frame}

\begin{frame}{Two types of parameter modifications}
  In a LP problem, 
\begin{eqnarray*}
 \text{max }\quad\uvec{c}^t \uvec{x}\\
 \text{subject to }\quad A\uvec{x}\leq \uvec{b}, \forall x_i\geq0
\end{eqnarray*}

one can have two situations: one may have interest in knowing what happens with modifications in the parameters $c$ or modifications in the parameters $b$.

In addition, one can explore what occurs when adding an extra constraint or adding a new variable.

\end{frame}

\begin{frame}{Case 1: sensitivity with respect to $c$}

  \begin{Exercise}
    A farmer wants to minimize the cost of the food given to her lifestock. Two different types of nutrients $A$ and $B$ are needed by the animals, and she needs a minimum nutrition to be achieved.
    \begin{center}
    \begin{tabular}{c|c|c|c}
     & A & B & price\\
     \hline
     Feed 1 & 10 & 3 & 16\\
     Feed 2 & 4 & 5 & 14\\
     \hline
     requirements & 124 & 60 & \\
    \end{tabular}
    \end{center}

    Find how resilient is she to the changes in the price? What happens with respect to basic vs non-basic variables in a general case?

  \end{Exercise}
\note{veure  \url{https://youtu.be/o1pznRt_-y0?t=3378} minut 50 i següent capítol}
\end{frame}

\begin{frame}{Case 2: sensitivity with respect to $b$}

  In such cases we will explore the coeficients of the objective function in the dual problem, instead.

  \begin{Exercise}
    Given the LP problem
    \begin{equation*}
      \begin{aligned}
        \text{minimize } \quad & z = 16x_1+14x_2 \\
        \text{subject to }\quad &
        \begin{array}{rcl}
          10x_1 + 4x_2 &\geq &124 \\
          3x_1 + 5x_2 &\geq &60 \\
          x_1,x_2 &\geq& 0
        \end{array}
      \end{aligned}
    \end{equation*}

    Explore the sensitivity of the minimum value of $z$ with respect to the parameters in the RHS of the contraints. Hint: consider the dual problem.

  \end{Exercise}

\end{frame}



\note{venim de \url{https://youtu.be/o1pznRt_-y0?t=1603} i \url{https://www.youtube.com/watch?v=yU8updOR87c}}

\begin{frame}{Shadow price is the solution of the dual problem}

  Solving the dual problem gives a lot of insight into the actual situation.

  \begin{Exercise}*
  In a company that manufactures two types of bikes, $A$ and $B$, the owners want to maximize the benefits, taking into account that the production depends on a restricted amount of titanium, the time the machines can devote to the work and the manpower, which are all given below:
  \begin{center}
  \begin{tabular}{c|c|c|c}
   & A & B & limit\\
   \hline
   Titanium & 50 & 30 & 2000\\
    Machine time & 6 & 5 & 300\\
    Labor & 3 & 5 &200\\
   \hline
   price & 50 & 60 & \\
  \end{tabular}
  \end{center}

  %Check the link https://app.wooclap.com/PDHGSS and answer the questions


  \end{Exercise}

\end{frame}


\section{Sensitivity Analysis}
\begin{frame}{Sensitivity Analysis: Overview}
  \begin{itemize}
    \item Study how optimal solutions to LPs change when input data change.
    \item Changes considered: objective coefficients, right-hand sides (resources), constraints.
    \item LP solutions can be \emph{very} sensitive — practical importance: decisions may change drastically.
    \item We will use a concrete example (Silicon Chip Corporation) to illustrate basic concepts:
      break-even prices, reduced costs, range analysis, marginal values, and the fundamental theorem of sensitivity analysis.
  \end{itemize}
\end{frame}

\section{Silicon Chip Corporation}

\begin{frame}
  \frametitle{Silicon Chip Corporation: Problem description}

  A Silicon Valley firm specializes in making four types of silicon chips for personal computers.
Each chip must go through four stages of processing before completion. First the basic silicon
wafers (100 chips each) are manufactured, second the wafers are laser etched with a micro circuit, next the circuit
is laminated onto the chip, and finally the chip is tested and packaged for shipping. The
production manager desires to maximize profits during the next month. During the next 30 days
she has enough raw material to produce 4000 silicon wafers. Moreover, she has 600 hours of
etching time, 900 hours of lamination time, and 700 hours of testing time. Taking into account
depreciated capital investment, maintenance costs, and the cost of labor, each raw silicon wafer
is worth \$1, each hour of etching time costs \$40, each hour of lamination time costs \$60, and
each hour of inspection time costs \$10.
The production manager has formulated her problem as a profit maximization

\end{frame}

\begin{frame}{Silicon Chip Corporation: Problem conceptualization}
  \begin{itemize}
    \item Four types of chips (types 1--4). Each batch is 100 chips.
    \item Four processing stages: raw wafers, etching, lamination, testing.
    \item Next 30 days availability:
      \begin{itemize}
        \item Raw wafers: 4000 units
        \item Etching: 600 hours
        \item Lamination: 900 hours
        \item Testing: 700 hours
      \end{itemize}
    \item Costs per resource (depreciated etc.): wafer \$1 each, etching \$40/hr, lamination \$60/hr, testing \$10/hr.
    \item Objective: maximize profit.
  \end{itemize}
\end{frame}

\begin{frame}{Decision variables and initial tableau}
  Let \(x_i\) = number of 100-chip batches of type \(i\) (for \(i=1,\dots,4\)). The initial tableau (coefficients per 100-chip batch):
  \[
  \begin{array}{c|cccc|cccc|c}
    & x_1 & x_2 & x_3 & x_4 & x_5 & x_6 & x_7 & x_8 & b\\
    \hline
    \text{raw wafers} & 100 & 100 & 100 & 100 & 1 & 0 & 0 & 0 & 4000\\
    \text{etching} & 10 & 10 & 20 & 20 & 0 & 1 & 0 & 0 & 600\\
    \text{lamination} & 20 & 20 & 30 & 20 & 0 & 0 & 1 & 0 & 900\\
    \text{testing} & 20 & 10 & 30 & 30 & 0 & 0 & 0 & 1 & 700\\
    \hline
    \text{profit (per 100)} & -2000 & -3000 & -5000 & -4000 & 0 & 0 & 0 & 0 & 0
  \end{array}
  \]
  (Here \(x_5,\dots,x_8\) are slack variables.)
\end{frame}

\begin{frame}{Optimal solution (given)}
  The (given) optimal tableau after simplex pivots is:
  \[
  \begin{array}{c|cccc|cccc|c}
    & x_1 & x_2 & x_3 & x_4 & x_5 & x_6 & x_7 & x_8 & b\\
    \hline
    & 0.5 & 1 & 0 & 0 & 0.015 & 0 & 0 & -0.05 & 25\\
    & -5 & 0 & 0 & 0 & -0.05 & 1 & 0 & -0.5 & 50\\
    & 0 & 0 & 1 & 0 & -0.02 & 0 & 0.1 & 0 & 10\\
    & 0.5 & 0 & 0 & 1 & 0.015 & 0 & -0.1 & 0.05 & 5\\
    \hline
    z & 1500 & 0 & 0 & 0 & 5 & 0 & 100 & 50 & 145000
  \end{array}
  \]
  \vspace{4pt}
  So the optimal production schedule in batches (100-chip units) is
  \[
    (x_1,x_2,x_3,x_4) = (0,25,10,5),
  \]
  with optimal profit \$145,000.
\end{frame}

\begin{frame}{Interpreting the objective-row entries}
  \begin{itemize}
    \item For a maximization problem in this tableau-sign convention, the objective row entries under nonbasic variables are \(\Delta = c - A^T y\) (up to sign convention). Negative entries indicate \emph{reduced costs}.
    \item Example: the entry under \(x_1\) is \(1500\) $\Rightarrow$ reduced cost $=1500$, so producing one 100-batch of type 1 would reduce \(z\) by \$1500.
    \item Break-even sale price adds this reduced cost to the current sale price to find the price at which the activity becomes profitable.
  \end{itemize}
\end{frame}


\section{Changing an objective coefficient: general pivot form}
\begin{frame}{Simplex pivots as left multiplication}
  The initial augmented matrix is
  \[
    \begin{bmatrix} A & I & b \\ c^{T} & 0 & 0 \end{bmatrix}.
  \]
  Pivoting to an optimal tableau corresponds to left-multiplication by a matrix
  \[
    G = \begin{bmatrix} R & 0 \\ -y^{T} & 1 \end{bmatrix},
  \]
  where \(R\) is the nonsingular record (basis) matrix and \(y\) is the dual vector.
  The resulting tableau is
  \[
    \begin{bmatrix}
      R A & R & R b\\
      c^{T} - y^{T} A & -y^{T} & -b^{T} y
    \end{bmatrix}.
  \]
  In the final tableau \(c - A^{T} y\) appears in the objective row.
\end{frame}


\begin{frame}{Effect of changing objective coefficients}
  If \(c\) changes to \(c + \Delta c\), performing the same left multiplication gives
  \[
    \begin{bmatrix}
      R A & R & R b\\
      (c + \Delta c)^{T} - y^{T} A & -y^{T} & -b^{T} y
    \end{bmatrix}
    =
    \begin{bmatrix}
      R A & R & R b\\
      (c - A^{T} y)^{T} + \Delta c^{T} & -y^{T} & -b^{T} y
    \end{bmatrix}.
  \]
  So the new objective-row entries are obtained by adding \(\Delta c\) to the old objective row.
  The current basis remains optimal if the modified objective-row entries keep the sign conditions (non-positive here).
\end{frame}

\section{Adding $\theta$ to $c_1$: $2000+\theta$ scenario}
\begin{frame}{Modify \(c_1\) to \(2000 + \theta\)}
  Replace the first objective coefficient by \(2000+\theta\). This corresponds to
  \(\Delta c = \theta e_1\). Applying the previous observation, the optimal tableau's objective row becomes
  \[
    \text{(old objective row)} + \theta e_1.
  \]
  Using the numeric optimal tableau earlier, the modified objective row (bottom row) has \(-1500 + \theta\) under \(x_1\).
  Thus to preserve optimality we require
  \[
    -1500 + \theta \le 0 \quad\Rightarrow\quad \theta \le 1500.
  \]
  Hence break-even occurs at \(\theta=1500\), matching the reduced-cost interpretation.
\end{frame}

\begin{frame}{Representative tableau with \(2000 + \theta\)}
  \[
  \begin{array}{c|cccc|cccc|c}
    & x_1 & x_2 & x_3 & x_4 & x_5 & x_6 & x_7 & x_8 & b\\
    \hline
    \text{rows} & 0.5 & 1 & 0 & 0 & 0.015 & 0 & 0 & -0.05 & 25\\
                & -5  & 0 & 0 & 0 & -0.05  & 1 & 0 & -0.5  & 50\\
                & 0   & 0 & 1 & 0 & -0.02  & 0 & 0.1 & 0     & 10\\
                & 0.5 & 0 & 0 & 1 & 0.015  & 0 & -0.1 & 0.05 & 5\\
    \hline
    z & -1500+\theta & 0 & 0 & 0 & -5 & 0 & -100 & -50 & 145000
  \end{array}
  \]
  Condition for current basis to remain optimal: \(-1500+\theta \le 0\) (and other objective-row entries non-positive, which they already are).
\end{frame}


\section{Break-even Prices and Reduced Costs}
\begin{frame}{Break-even price — intuition}
  \begin{itemize}
    \item A decision variable not in the basis (nonbasic) has an \textbf{objective row coefficient} in the final tableau.
    \item The \textbf{reduced cost} of a nonbasic variable equals the negative of that objective-row coefficient (when objective expressed as \(z - (\cdots)=0\)).
    \item Reduced cost = how much the objective coefficient must increase before the variable enters the basis.
  \end{itemize}
\end{frame}

\begin{frame}{Example: Type 1 chip break-even price}
  Current per-batch profit shown in original LP: \$2000 per 100-unit batch for type 1.
  Compute production cost per 100-batch:
  \[
  \begin{aligned}
    \text{chip cost} &= 100\times 1 = 100\\
    \text{etching cost} &= 10\times 40 = 400\\
    \text{lamination cost} &= 20\times 60 = 1200\\
    \text{inspection cost} &= 20\times 10 = 200\\
    \text{total cost per batch} &= 1900
  \end{aligned}
  \]
  So current sale price per batch = profit + cost = \(2000 + 1900 = \$3900\), i.e. \$39 per chip.
\end{frame}

\begin{frame}{Type 1 reduced cost and break-even}
  The reduced cost of a decision variable is the needed increase in its objective row coefficient in
order for it to be included in the optimal solution.
For non-basic variables the break-even sale price can be read off from the reduced costs in the
optimal tableau.

  In the optimal tableau the objective row entry under \(x_1\) equals \(-1500\) (objective row entry is \(-1500\) in the printed tableau).
  That means producing one batch of type 1 reduces optimal \(z\) by \$1500.
  To make \(x_1\) profitable, the sale price must increase by \$1500 per 100-batch, i.e. \$15 per chip.
  So break-even sale price per chip \(= 39 + 15 = \$54\).
\end{frame}

\begin{frame}
  \frametitle{More intuitive explanation}

  Now consider a more intuitive and simpler explanation of break-even sale prices.
One way to determine these prices, is to determine by how much our profit is reduced if
we produce one batch of these chips.
Recall that the objective row coefficients in the optimal tableau correspond to the
following expression for the objective variable z:
z = 145000 − 1500x1 − 5x5 − 100x7 − 50x8 .
Hence, if we make one batch of type 1 chip, we reduce our optimal value by \$1500.
Thus, to recoup this loss we must charge \$1500 more for these chips yielding a
break-even sale price of \$39 + \$15 = \$54 per chip.

\end{frame}

\section{Range Analysis for Objective Coefficients}
\begin{frame}{Range analysis: goal}
  Range analysis is a tool for understanding the effects of both objective coefficient
variations as well as resource availability variations.
  \begin{itemize}
    \item For a basic variable, find interval of objective coefficient values for which the current basis stays optimal.
    \item For a nonbasic variable, find how much objective coefficient can change before it becomes profitable to enter the basis.
  \end{itemize}
\end{frame}

\begin{frame}{Range analysis: method (summary)}
  \begin{itemize}
    \item Changing \(c\) (objective coefficients) to \(c+\Delta c\) results in adding \(\Delta c\) to the objective row of the current tableau after applying the same pivot record matrix.
    \item If the modified objective row (for nonbasic columns) remains non-positive (for a maximization LP in this tableau convention), the current basis remains optimal.
    \item Solve inequalities on the modified objective row entries to get allowable ranges.
  \end{itemize}
\end{frame}

\begin{frame}{Range for \(c_3\) (type 3 chips)}
  Let \(c_3 = 5000 + \theta\). The objective row after adjusting gives (reduced) modifications that must satisfy:
  \[
    \begin{aligned}
      -5 + 0.02\theta &\le 0 \quad\Rightarrow\quad \theta \le 250,\\
      -100 - 0.1\theta &\le 0 \quad\Rightarrow\quad \theta \ge -1000.
    \end{aligned}
  \]
  Thus
  \[
    -1000 \le \theta \le 250 \quad\Rightarrow\quad 4000 \le c_3 \le 5250.
  \]
  So the basis \(\{x_2,x_3,x_4,x_6\}\) remains optimal while \(c_3\) stays in this interval.
\end{frame}

\begin{frame}{Perturb \(c_3 = 5000 + \theta\) (type 3 in basis)}
  Start with \(c_3 = 5000+\theta\). After adding \(\theta e_3\) to the objective row of the current tableau, we obtain a tableau with an extra \(\theta\) under column \(x_3\) in the objective row:
  \[
  \begin{array}{c|cccc|cccc|c}
    & x_1 & x_2 & x_3 & x_4 & x_5 & x_6 & x_7 & x_8 & b\\
    \hline
    \text{(rows)} & 0.5 & 1 & 0 & 0 & 0.015 & 0 & 0 & -0.05 & 25\\
                  & -5  & 0 & 0 & 0 & -0.05  & 1 & 0 & -0.5  & 50\\
                  & 0   & 0 & 1 & 0 & -0.02  & 0 & 0.1 & 0     & 10\\
                  & 0.5 & 0 & 0 & 1 & 0.015  & 0 & -0.1 & 0.05 & 5\\
    \hline
    z & -1500 & 0 & \theta & 0 & -5 & 0 & -100 & -50 & 145000
  \end{array}
  \]
  This is not a proper tableau (because of nonzero objective entry under a basic column); we must pivot to eliminate \(\theta\) (i.e., adjust by multiples of the basic row corresponding to \(x_3\)).
\end{frame}

\begin{frame}{Eliminate $\theta$ by adding \(-\theta\) times row 3 to the objective row}
  Multiply row 3 (the basic row for \(x_3\)) by \(-\theta\) and add to the objective row:
  \[
    z_{\text{new}} = z_{\text{old}} - \theta\cdot\text{row}_3.
  \]
  This changes several objective-row entries; with the numeric row entries we get (representative):
  \[
  \begin{array}{c|cccc|cccc|c}
    \hline
    z & -1500 & 0 & 0 & 0 & -5 + 0.02\theta & 0 & -100-0.1\theta & -50 & 145000 - 10\theta
  \end{array}
  \]
  To preserve optimality all entries that correspond to nonbasic variables must remain non-positive:
  \[
    -5 + 0.02\theta \le 0,\quad -100 - 0.1\theta \le 0,\quad \dots
  \]
  Solving yields:
  \[
    -1000 \le \theta \le 250 \quad\Rightarrow\quad 4000 \le c_3 \le 5250.
  \]
\end{frame}


\begin{frame}{Range for \(c_4\) (type 4 chips) and \(c_2\)}
  A similar perturbation for \(c_4=4000+\theta\) yields inequalities (from objective row entries) which give:
  \[
    -333.\overline{3} \le \theta \le 1000,
  \]
  so
  \[
    3666.\overline{6} \le c_4 \le 5000.
  \]
  For \(c_2\) (originally 3000) the derived range is
  \[
    1666.\overline{6} \le c_2 \le 3000.
  \]
  (These follow from ensuring all objective-row entries remain non-positive.)
\end{frame}

\begin{frame}{Perturb \(c_4 = 4000 + \theta\)}
  A similar calculation for \(c_4\) yields (after appropriate row operations and algebra) the inequalities:
  \[
    -333.\overline{3} \le \theta \le 1000,
  \]
  giving
  \[
    3666.\overline{6} \le c_4 \le 5000.
  \]
  (These bounds come from ensuring all objective-row entries remain non-positive after the adjustments.)
\end{frame}

\begin{frame}{Range for \(c_2\) (summary)}
  For \(c_2\) the lecture's algebra gives:
  \[
    -333.\overline{3} \le \theta \le 1000 \quad\Rightarrow\quad 1666.\overline{6} \le c_2 \le 3000
  \]
  (Again, derived from the sign constraints on the objective row after adding \(\Delta c\) and pivoting as needed.)
\end{frame}


\section{Resource Variations, Marginal Values, and Range Analysis}
\begin{frame}{Resource variation questions}
  Suppose we change a right-hand-side (RHS) resource (e.g., purchase more wafers). Typical questions:
  \begin{enumerate}
    \item How many additional units should we buy?
    \item What is the most we should pay per unit?
    \item What is the new optimal schedule after purchasing?
  \end{enumerate}
  The dual variables (shadow prices) provide the marginal value of resources for small changes (within allowable intervals).
\end{frame}

\begin{frame}{RHS perturbation and marginal values (sketch)}
  \begin{itemize}
    \item Let \(b\) be the RHS vector and suppose basis \(B\) is optimal with record matrix \(R\).
    \item The basic solution is \(x_B = R b\). If \(b\) changes by \(\Delta b\), the basic solution becomes \(x_B = R (b + \Delta b)\).
    \item If \(x_B\) remains feasible (nonnegative), the basis remains feasible; combined with dual feasibility, the basis remains optimal.
    \item The dual solution \(y\) remains the same for small enough changes; the change in optimal objective is \(y^\top \Delta b\).
  \end{itemize}
\end{frame}

\section{Right Hand Side Perturbations}
\begin{frame}{Allowable RHS changes (concept)}
  \begin{itemize}
    \item For each basic variable, determine how far the RHS can move until a basic variable becomes zero (i.e., feasibility violated).
    \item These bounds yield intervals for each RHS entry within which the current basis remains optimal.
    \item Shadow prices hold within these intervals.
  \end{itemize}
\end{frame}

\begin{frame}{Example: marginal value interpretation}
  If the dual variable (shadow price) associated with raw wafers is \(\pi_1\), then:
  \[
    \text{small increase of 1 wafer} \Rightarrow \text{increase in objective} \approx \pi_1.
  \]
  For finite changes, use range analysis to confirm how large a change preserves the basis (so \(\pi_1\) stays valid).
\end{frame}



\section{Pricing Out}
\begin{frame}{Pricing out (brief)}
  \begin{itemize}
    \item When a resource is extremely expensive (or cheap) it may be optimal to remove it (or saturate it).
    \item Analyze by varying the RHS or adding cost to resource availability to see when shadow prices change sign or basis changes.
  \end{itemize}
\end{frame}

\section{The Fundamental Theorem on Sensitivity Analysis}
\begin{frame}{Fundamental theorem (informal statement)}
  Given a linear program with current optimal basis \(B\), there exist non-empty intervals for each objective coefficient and each RHS entry such that any changes to those coefficients within their respective intervals preserve the optimal basis.
  Within those intervals:
  \begin{itemize}
    \item The dual solution stays constant.
    \item The objective value changes linearly with the perturbations (via the dual for RHS, via basic solution for objective-row changes).
  \end{itemize}
\end{frame}

\begin{frame}{Practical takeaways}
  \begin{itemize}
    \item Reduced costs read directly from the final tableau give break-even adjustments for nonbasic variables.
    \item Adding \(\Delta c\) to \(c\) simply adds \(\Delta c\) to the objective row of the final tableau (then pivot if necessary).
    \item To analyze RHS changes use \(x_B = R b\) and dual vector \(y\); objective change \(\Delta z = y^\top \Delta b\) while the basis remains valid.
  \end{itemize}
\end{frame}

\begin{frame}{Practical takeaways}
  \begin{itemize}
    \item Use reduced costs to read break-even prices for nonbasic variables.
    \item Use range analysis to find intervals where the basis remains optimal (both for \(c\) and \(b\)).
    \item Use shadow prices (dual variables) for marginal valuations of resources — valid for small changes within allowable intervals.
    \item When changes fall outside allowable intervals, recompute basis (re-run LP) or update tableau with pivots.
  \end{itemize}
\end{frame}

\begin{frame}{Some tools that help interpreting the calculations}

  Find \href{https://github.com/Biocomputing-Teaching/ORcourse/blob/main/code/linear_programming.ipynb}{\em here} a code with the solution using ORtools.
  \\[10pt]
  The Sensitivity analysis tool in Excel is nicely explained \href{https://www.youtube.com/watch?v=zKqU5NGE-t0&list=PLjiMsqjDUvBiArYMqCGZSDJNgEMPALdQu&index=1}{\em here}.
\\[10pt]
  Notice that in a \href{https://youtu.be/v2ZuKwAaXOI}{non-linear scenario}, what we have identified here as {\em reduced costs} are in fact {\em reduced gradient} (or actual gradient of the optimal solution) while {\em shadow prices} (margin for profit to be obtained by changing 1 unit in each constraint RHS term) correspond to the {\em Lagrange multipliers}.

\end{frame}


\section*{References}
\begin{frame}{References}
    \footnotesize
    \begin{thebibliography}{99}
    \setbeamertemplate{bibliography item}[text]
      \begin{columns}[t]
        \begin{column}{.45\textwidth}
            \bibitem{carter} Michael W. Carter, Camille C. Price, and Ghaith Rabadi. Operations Research, 2nd Edition. CRC Press.
            \bibitem{harel} David Harel, with Yishai Feldman. Algorithmics: the spirit of computing, 3rd Edition. Addison-Wesley.
            \bibitem{rardin} Ronald L. Rardin. Optimization in Operations Research, 2nd Edition. Pearson.
            \bibitem{hefferon} J. Hefferon. \href{http://joshua.smcvt.edu/linearalgebra}{Linear algebra (4th Ed)}.
        \end{column}
        \begin{column}{.45\textwidth}
            \bibitem{riley} K.F. Riley, M.P. Hobson, S.J. Bence. Mathematical Methods for Physics and Engineering (2nd Ed). McGraw Hill.
            \bibitem{nocedal} J. Nocedal, S. J. Wright. Numerical Optimization (2nd Ed). Springer.
            \bibitem{beers} Kenneth J. Beers. Numerical methods for chemical engineering: applications in Matlab. Cambridge University Press.
            \bibitem{barber} D. Barber. Bayesian reasoning and machine learning. Cambridge University Press.
        \end{column}
      \end{columns}
    \end{thebibliography}
\end{frame}
%----------------------------------------------------------------------------------------

\end{document}
