%%%%%%%%%%%%%%%%%%%%%%%%%%%%%%%%%%%%%%%%%
% Beamer Presentation
% LaTeX Template
% Version 1.0 (10/11/12)
%
% This template has been downloaded from:
% http://www.LaTeXTemplates.com
%
% License:
% CC BY-NC-SA 3.0 (http://creativecommons.org/licenses/by-nc-sa/3.0/)
%
%%%%%%%%%%%%%%%%%%%%%%%%%%%%%%%%%%%%%%%%%

%----------------------------------------------------------------------------------------
%	PACKAGES AND THEMES
%----------------------------------------------------------------------------------------

\documentclass[c]{beamer}
%\documentclass[notes]{beamer}
\setbeamertemplate{note page}[show only notes]
\input{../OR_common.tex}
%%%%%%%%%%%%%%%%%%%%%%%%%%%%%%%%%%%%%%%%%%%%%%%%%%%%%%%%%%%%%%%%%%%%%%%%%%%%%
%%%%%%%%%%%%%%%%%%%%%%%%%%%%%%%%%%%%%%%%%%%%%%%%%%%%%%%%%%%%%%%%%%%%%%%%%%%%%
%%%%%%%%%%%%%%%%%%%%%%%%%%%%%%%%%%%%%%%%%%%%%%%%%%%%%%%%%%%%%%%%%%%%%%%%%%%%%

\title[Introduction]{Unit 2. Linear programming. The Simplex Method}

\author{Jordi Villà i Freixa}
\institute[FCTE]{
Universitat de Vic - Universitat Central de Catalunya \\
Study Abroad. Operations Research\\
\medskip
\textit{jordi.villa@uvic.cat}
}
\date{28/03-18/04, 2023}
\logo{\includegraphics[width=.1\textwidth]{FCTE}}
\begin{document}

\begin{frame}
\titlepage
\end{frame}

\begin{frame}
\frametitle{Preliminary}
This course is strongly based on the monography on Operations Research by Carter, Price and Rabadi \cite{carter}, and in material obtained from different sources (quoted when needed through the slides).
\end{frame}

%%%%%%%%%%%%%%%%%%%%%%%%%%%%%%%%%%%%%%%%%%%%%%%%%%%%%%%%%%%%%%%%%%%%%%%%%%%%%
%%%%%%%%%%%%%%%%%%%%%%%%%%%%%%%%%%%%%%%%%%%%%%%%%%%%%%%%%%%%%%%%%%%%%%%%%%%%%
%%%%%%%%%%%%%%%%%%%%%%%%%%%%%%%%%%%%%%%%%%%%%%%%%%%%%%%%%%%%%%%%%%%%%%%%%%%%%

\begin{frame}
\frametitle{Learning outcomes}
\begin{itemize}
  \item Understanding the rational behind the Simplex method for LP.
  \item Understanding and practicing the algorithm in 2 variables.
  \item Recognizing the different types of results one can achieve in LP from the Simplex Method algorithm.
  \item Getting familiar with slack, surplus and artificial variables.
\end{itemize}
\end{frame}

\section{Preparing the LP problem for applying the Simplex method}

\begin{frame}{The standard form}

  \begin{block}{Standard form}
    For a LP with $n$ variables and $m$ constraints, the standard form is given by:
    \begin{equation*}
      \begin{aligned}
        \text{maximize } \quad & z = c_1x_1+c_2x_2 +\cdots + c_nx_n \\
        \text{subject to }\quad &
        \begin{array}{rcl}
          a_{11}x_1+a_{12}x_2+\cdots+a_{1n}x_n &= &b_1 \\
          a_{21}x_1+a_{22}x_2+\cdots+a_{2n}x_n &= &b_2 \\
          \vdots &=& \vdots\\
          a_{m1}x_1+a_{m2}x_2+\cdots+a_{mn}x_n &= &b_m \\
        \end{array}
      \end{aligned}
    \end{equation*}
    where $x_1,\ldots,x_n\geq0$ and $b_1,\ldots,b_m\geq0$
  \end{block}

Thus:
\begin{equation*}
  \begin{aligned}
    \text{maximize } \quad & z = cx \\
    \text{subject to }\quad &
    \begin{array}{c}
      Ax=b\\
      x\geq0\\
      b\geq0
    \end{array}
  \end{aligned}
\end{equation*}

\end{frame}

\begin{frame}
Of course, not always the system is proposed in this standard form, so:
\begin{itemize}
  \item We need to leave a maximization problem. If the problem is to minimize the objective function, we simply multiply it by $(-1)$.
  \item In order to change inequality constraints into equality constraints we use {\it slack} variables for $\leq$ inequalities:
  \[
  3x_1+4x_2 \leq 7  \rightarrow  3x_1+4x_2+s_1 =7
  \]
  or {\it surplus} variables for $\geq$ inequalities:
  \[
  x_1+3x_2\geq 10 \rightarrow x_1+3x_2 -s_2 =10
  \]
  \item All variables should be non-negative.
\end{itemize}
\end{frame}

\begin{frame}{The algorithm}
\begin{enumerate}
  \item Convert the system of inequalities to equations (using slack or surplus variables).
  \item Set the objective function to zero.
  \item Create the Simplex tableau and label active and basic variables.
  \item Select the pivot column (the one with the most negative coefficient in the zeroed objective function). This will be linked to the {\bf entering vaiable}.
  \item Select the pivot row (once divided the entry in the constant column by the ceffient in that row in the pivot column, we choose the smallest ratio). This will be linked to the {\bf leaving variable}.
  \item The pivot is the intersection between the pivot row and pivot column.
  \item Use the pivot value to make zeros in the rest of elements in the pivot column.
  \item Repeat the process from step 4, until the last row is all non-negative.
\end{enumerate}

\end{frame}

\begin{frame}{Example. Standard form}


  \begin{equation*}
    \begin{aligned}
      \text{maximize } \quad & z = 8x_1+5x_2 \\
      \text{subject to }\quad &
      \begin{array}{rcl}
        x_1 &\leq &150 \\
        x_2 &\leq &250 \\
        2x_1+x_2 &\leq &500 \\
        x_1,x_2 &\geq& 0
      \end{array}
    \end{aligned}
  \end{equation*}
We build first the standard form:
\begin{eqnarray*}
  -8x_1-5x_2+z&=&0\\
  x_1+s&=&150\\
  x_2+t&=&250\\
  2x_1+x_2+u&=&500
\end{eqnarray*}
\end{frame}

\begin{frame}{Example. The Simplex Tableau}
  We write the coefficients matrix and we identify the basic ($m=3$) and non-basic ($n-m=5-3=2$) variables:
  \begin{equation}
\begin{array}{cc}
&\\
&z \\
\rightarrow &s \\
&t \\
&u\\
\mathrm{basic}
\end{array}
%
\begin{array}{c|ccccc|c}
  z & x_1 & x_2 & s & t & u & b \\ \hline
  1 & -8 & -5 & 0 & 0 & 0 & 0 \\ \hline
  0 & 1 & 0 & 1 & 0 & 0 & 150  \\
  0 & 0 & 1 & 0 & 1 & 0 & 250 \\
  0 & 2 & 1 & 0 & 0 & 1 & 500 \\
    & \uparrow & & & & &
\end{array}
%\end{matrix}
\end{equation}
  Note that the basic variables are those for which each column is a collection of 1 and zero. We will arbitrarily assign zeros to the non-basic variables.
So a possible solution is $\boxed{P_A=(x_1=0, x_2=0)\Rightarrow z=0}$.

  The arrow marks the pivot column. We will take the pivot row by considering which is the lowest value among 150/1 and 500/2. So, the pivot row corresponds to the $s$ basic variable.
\end{frame}

\begin{frame}{Example. Graphic solution}
  \begin{center}
   \includegraphics[width=0.5\linewidth]{LP7.pdf}
  \end{center}
\end{frame}

\begin{frame}{Example. Entering/leaving variables}

\begin{equation*}
\begin{array}{cc}
&\\
R_1+8R_2&z \\
&x_1 \\
&t \\
\rightarrow R_4-2R_2&u\\
&\mathrm{basic} \\
\end{array}
\begin{array}{c|ccccc|c}
  z & x_1 & x_2 & s & t & u & b \\ \hline
  1 & 0 & -5 & 8 & 0 & 0 & 1200 \\ \hline
  0 & 1 & 0 & 1 & 0 & 0 & 150  \\
  0 & 0 & 1 & 0 & 1 & 0 & 250 \\
  0 & 0 & 1 & -2 & 0 & 1 & 200 \\
    &  & \uparrow& & & &
\end{array}
\end{equation*}
$P_B=(150,0)$ and $z(P_B)=1200$ with $t=250$, $u=200$, $s=0$.
\end{frame}

\begin{frame}
\begin{equation*}
\begin{array}{cc}
&\\
R_1+5R_4&z \\
&x_1 \\
\rightarrow R_3-R_4&t \\
&x_2\\
&\mathrm{basic} \\
\end{array}
\begin{array}{c|ccccc|c}
  z & x_1 & x_2 & s & t & u & b \\ \hline
  1 & 0 & 0 & -2 & 0 & 5 & 2200 \\ \hline
  0 & 1 & 0 & 1 & 0 & -1 & 150  \\
  0 & 0 & 0 & 2 & 0 & 1 & 50 \\
  0 & 0 & 1 & -2 & 0 & 1 & 200 \\
    &  & & \uparrow& & &
\end{array}
\end{equation*}

\begin{equation*}
\begin{array}{cc}
&\\
R_1+5R_4&z \\
&x_1 \\
\rightarrow R_3-R_4&s \\
&x_2\\
&\mathrm{basic} \\
\end{array}
\begin{array}{c|ccccc|c}
  z & x_1 & x_2 & s & t & u & b \\ \hline
  1 & 0 & 0 & 0 & 1 & 4 & 2250 \\ \hline
  0 & 1 & 0 & 0 & -1/2 & 1/2 & 125  \\
  0 & 0 & 0 & 1 & 1/2 & -1/2 & 25 \\
  0 & 1 & 0 & 1 & 0 & 0 & 250 \\
    &  & & \uparrow& & &
\end{array}
\end{equation*}

$\boxed{P_D=(125,250)}$ and $\boxed{z(P_D)=2250}$ with $t,u=0$ (binding constraints), $s=25$ (non-binding constraint).
\end{frame}

\begin{frame}{Exercises}
  Solve, using the Simplex method, the Exercises 3 (p17) and 6 (p23) of the previous session
\end{frame}


\begin{frame}{References}
\footnotesize
\begin{thebibliography}{99} % Beamer does not support BibTeX so references must be inserted manually as below
  \begin{columns}[t]
    \begin{column}{.45\textwidth}
      \bibitem[Carter, 2019]{carter} Michael W. Carter, Camille C. Price, and Ghaith Rabadi (2019)
        \newblock Operations Research, 2nd Edition
        \newblock \emph{CRC Press}.
        \bibitem[Harel, 2019]{harel} David Harel, with Yishai Feldman (2004)
          \newblock Algorithmics: the spirit of computing, 3rd Edition
          \newblock \emph{Addison-Wesley}.
    \end{column}
    \begin{column}{.45\textwidth}
      \bibitem[Riley, 2002]{riley} K.F. Riley, M.P. Hobson, S.J. Bence (2002)
        \newblock Mathematical Methods for Physics and Engineering (2nd Ed)
        \newblock \emph{McGraw Hill}.
      \bibitem[Nocedal, 2006]{nocedal} J. Nocedal, S. J. Wright (2006)
        \newblock Numerical Optimization (2nd Ed)
        \newblock \emph{Springer}.
    \end{column}
  \end{columns}
\end{thebibliography}
\end{frame}
%----------------------------------------------------------------------------------------

\end{document}
