% L13-sensitivity-with-tableaux.tex
\documentclass{beamer}
\usetheme{Madrid}
\usecolortheme{dolphin}
\usepackage{amsmath,amssymb,booktabs,caption}
\usepackage{array}
\usepackage{siunitx}
\usepackage{graphicx}
\setbeamertemplate{caption}{\raggedright\insertcaption\par}
\newcommand{\matrow}[1]{\rule{0pt}{2.6ex}#1}
\newcolumntype{C}{>{$}c<{$}}

\title[Lecture 13]{Lecture 13: Sensitivity Analysis}
\subtitle{Linear Programming — Representative Tableaux Included}
\author{Department of Industrial Engineering}
\institute{Course: Linear Programming}
\date{}

\begin{document}

\begin{frame}
  \titlepage
\end{frame}

\begin{frame}{Contents}
  \tableofcontents
\end{frame}

\section{Problem setup and initial tableau}
\begin{frame}{Silicon Chip Corporation (recap)}
  \begin{itemize}
    \item Decision variables: \(x_i\) = number of 100-chip batches of type \(i=1,\dots,4\).
    \item Resources and availability: wafers 4000, etching 600 hrs, lamination 900 hrs, testing 700 hrs.
    \item Profit per 100-batch: \(c = (2000,3000,5000,4000)^T\).
    \item Slack variables \(x_5,\dots,x_8\).
  \end{itemize}
\end{frame}

\begin{frame}{Initial simplex tableau (compact)}
  \[
  \begin{array}{c|cccc|cccc|c}
    & x_1 & x_2 & x_3 & x_4 & x_5 & x_6 & x_7 & x_8 & b\\
    \hline
    \text{raw wafers} & 100 & 100 & 100 & 100 & 1 & 0 & 0 & 0 & 4000\\
    \text{etching}    & 10  & 10  & 20  & 20  & 0 & 1 & 0 & 0 & 600\\
    \text{lamination} & 20  & 20  & 30  & 20  & 0 & 0 & 1 & 0 & 900\\
    \text{testing}    & 20  & 10  & 30  & 30  & 0 & 0 & 0 & 1 & 700\\
    \hline
    z\text{-row (profit)} & 2000 & 3000 & 5000 & 4000 & 0 & 0 & 0 & 0 & 0
  \end{array}
  \]
  (This reproduces the initial augmented tableau of the LP.)
\end{frame}

\section{Optimal tableau and interpretation}
\begin{frame}{Given optimal tableau (representative form)}
  The lecture gives the following optimal tableau (numbers shown are the essential entries):
  \[
  \begin{array}{c|cccc|cccc|c}
    & x_1 & x_2 & x_3 & x_4 & x_5 & x_6 & x_7 & x_8 & b\\
    \hline
    r_1 & 0.5 & 1 & 0 & 0 & 0.015 & 0 & 0 & -0.05 & 25\\
    r_2 & -5  & 0 & 0 & 0 & -0.05  & 1 & 0 & -0.5  & 50\\
    r_3 & 0   & 0 & 1 & 0 & -0.02  & 0 & 0.1 & 0     & 10\\
    r_4 & 0.5 & 0 & 0 & 1 & 0.015  & 0 & -0.1 & 0.05 & 5\\
    \hline
    z   & -1500 & 0 & 0 & 0 & -5 & 0 & -100 & -50 & 145000
  \end{array}
  \]
  From this we read the optimal basic solution
  \((x_1,x_2,x_3,x_4)=(0,25,10,5)\) and optimal objective \(z=145000\).
\end{frame}

\begin{frame}{Interpreting the objective-row entries}
  \begin{itemize}
    \item For a maximization problem in this tableau-sign convention, the objective row entries under nonbasic variables are \(\Delta = c - A^T y\) (up to sign convention). Negative entries indicate \emph{reduced costs}.
    \item Example: the entry under \(x_1\) is \(-1500\) $\Rightarrow$ reduced cost $=1500$, so producing one 100-batch of type 1 would reduce \(z\) by \$1500.
    \item Break-even sale price adds this reduced cost to the current sale price to find the price at which the activity becomes profitable.
  \end{itemize}
\end{frame}

\section{Changing an objective coefficient: general pivot form}
\begin{frame}{Simplex pivots as left multiplication}
  The initial augmented matrix is
  \[
    \begin{bmatrix} A & I & b \\ c^{T} & 0 & 0 \end{bmatrix}.
  \]
  Pivoting to an optimal tableau corresponds to left-multiplication by a matrix
  \[
    G = \begin{bmatrix} R & 0 \\ -y^{T} & 1 \end{bmatrix},
  \]
  where \(R\) is the nonsingular record (basis) matrix and \(y\) is the dual vector.
  The resulting tableau is
  \[
    \begin{bmatrix}
      R A & R & R b\\
      c^{T} - y^{T} A & -y^{T} & -b^{T} y
    \end{bmatrix}.
  \]
  In the final tableau \(c - A^{T} y\) appears in the objective row.
\end{frame}

\begin{frame}{Effect of changing objective coefficients}
  If \(c\) changes to \(c + \Delta c\), performing the same left multiplication gives
  \[
    \begin{bmatrix}
      R A & R & R b\\
      (c + \Delta c)^{T} - y^{T} A & -y^{T} & -b^{T} y
    \end{bmatrix}
    =
    \begin{bmatrix}
      R A & R & R b\\
      (c - A^{T} y)^{T} + \Delta c^{T} & -y^{T} & -b^{T} y
    \end{bmatrix}.
  \]
  So the new objective-row entries are obtained by adding \(\Delta c\) to the old objective row.
  The current basis remains optimal if the modified objective-row entries keep the sign conditions (non-positive here).
\end{frame}

\section{Adding $\theta$ to $c_1$: $2000+\theta$ scenario}
\begin{frame}{Modify \(c_1\) to \(2000 + \theta\)}
  Replace the first objective coefficient by \(2000+\theta\). This corresponds to
  \(\Delta c = \theta e_1\). Applying the previous observation, the optimal tableau's objective row becomes
  \[
    \text{(old objective row)} + \theta e_1.
  \]
  Using the numeric optimal tableau earlier, the modified objective row (bottom row) has \(-1500 + \theta\) under \(x_1\).
  Thus to preserve optimality we require
  \[
    -1500 + \theta \le 0 \quad\Rightarrow\quad \theta \le 1500.
  \]
  Hence break-even occurs at \(\theta=1500\), matching the reduced-cost interpretation.
\end{frame}

\begin{frame}{Representative tableau with \(2000 + \theta\)}
  \[
  \begin{array}{c|cccc|cccc|c}
    & x_1 & x_2 & x_3 & x_4 & x_5 & x_6 & x_7 & x_8 & b\\
    \hline
    \text{rows} & 0.5 & 1 & 0 & 0 & 0.015 & 0 & 0 & -0.05 & 25\\
                & -5  & 0 & 0 & 0 & -0.05  & 1 & 0 & -0.5  & 50\\
                & 0   & 0 & 1 & 0 & -0.02  & 0 & 0.1 & 0     & 10\\
                & 0.5 & 0 & 0 & 1 & 0.015  & 0 & -0.1 & 0.05 & 5\\
    \hline
    z & -1500+\theta & 0 & 0 & 0 & -5 & 0 & -100 & -50 & 145000
  \end{array}
  \]
  Condition for current basis to remain optimal: \(-1500+\theta \le 0\) (and other objective-row entries non-positive, which they already are).
\end{frame}

\section{Break-even price: numeric interpretation}
\begin{frame}{Break-even price computed numerically}
  \begin{itemize}
    \item Current sale price per 100-batch for type 1: \$3900 (profit \$2000 + cost \$1900).
    \item Reduced cost per 100-batch: \$1500 (from objective row entry).
    \item Break-even sale price per 100-batch: \(3900 + 1500 = \$5400\), i.e. \$54 per chip.
  \end{itemize}
  (Matches the lecture conclusion.)
\end{frame}

\section{Perturb \(c_3\): range analysis for a basic variable}
\begin{frame}{Perturb \(c_3 = 5000 + \theta\) (type 3 in basis)}
  Start with \(c_3 = 5000+\theta\). After adding \(\theta e_3\) to the objective row of the current tableau, we obtain a tableau with an extra \(\theta\) under column \(x_3\) in the objective row:
  \[
  \begin{array}{c|cccc|cccc|c}
    & x_1 & x_2 & x_3 & x_4 & x_5 & x_6 & x_7 & x_8 & b\\
    \hline
    \text{(rows)} & 0.5 & 1 & 0 & 0 & 0.015 & 0 & 0 & -0.05 & 25\\
                  & -5  & 0 & 0 & 0 & -0.05  & 1 & 0 & -0.5  & 50\\
                  & 0   & 0 & 1 & 0 & -0.02  & 0 & 0.1 & 0     & 10\\
                  & 0.5 & 0 & 0 & 1 & 0.015  & 0 & -0.1 & 0.05 & 5\\
    \hline
    z & -1500 & 0 & \theta & 0 & -5 & 0 & -100 & -50 & 145000
  \end{array}
  \]
  This is not a proper tableau (because of nonzero objective entry under a basic column); we must pivot to eliminate \(\theta\) (i.e., adjust by multiples of the basic row corresponding to \(x_3\)).
\end{frame}

\begin{frame}{Eliminate $\theta$ by adding \(-\theta\) times row 3 to the objective row}
  Multiply row 3 (the basic row for \(x_3\)) by \(-\theta\) and add to the objective row:
  \[
    z_{\text{new}} = z_{\text{old}} - \theta\cdot\text{row}_3.
  \]
  This changes several objective-row entries; with the numeric row entries we get (representative):
  \[
  \begin{array}{c|cccc|cccc|c}
    \hline
    z & -1500 & 0 & 0 & 0 & -5 + 0.02\theta & 0 & -100-0.1\theta & -50 & 145000 - 10\theta
  \end{array}
  \]
  To preserve optimality all entries that correspond to nonbasic variables must remain non-positive:
  \[
    -5 + 0.02\theta \le 0,\quad -100 - 0.1\theta \le 0,\quad \dots
  \]
  Solving yields:
  \[
    -1000 \le \theta \le 250 \quad\Rightarrow\quad 4000 \le c_3 \le 5250.
  \]
\end{frame}

\section{Perturb \(c_4\): range analysis example}
\begin{frame}{Perturb \(c_4 = 4000 + \theta\)}
  A similar calculation for \(c_4\) yields (after appropriate row operations and algebra) the inequalities:
  \[
    -333.\overline{3} \le \theta \le 1000,
  \]
  giving
  \[
    3666.\overline{6} \le c_4 \le 5000.
  \]
  (These bounds come from ensuring all objective-row entries remain non-positive after the adjustments.)
\end{frame}

\begin{frame}{Range for \(c_2\) (summary)}
  For \(c_2\) the lecture's algebra gives:
  \[
    -333.\overline{3} \le \theta \le 1000 \quad\Rightarrow\quad 1666.\overline{6} \le c_2 \le 3000
  \]
  (Again, derived from the sign constraints on the objective row after adding \(\Delta c\) and pivoting as needed.)
\end{frame}

\section{RHS perturbations and shadow prices (sketch)}
\begin{frame}{RHS perturbations and marginal values}
  \begin{itemize}
    \item Let \(b\) be RHS vector and \(x_B = R b\) the basic solution. If \(b\) changes by \(\Delta b\), the basic solution moves to \(x_B = R(b+\Delta b)=x_B + R\Delta b\).
    \item If \(R\Delta b\) keeps \(x_B\) nonnegative, the basis remains feasible; combined with unchanged dual feasibility, the basis remains optimal.
    \item The change in objective (while the dual stays valid) is \(y^{T}\Delta b\) (linear in \(\Delta b\)).
    \item Thus \(y\) (the dual/shadow prices) gives marginal value of resources for small changes inside allowable RHS ranges.
  \end{itemize}
\end{frame}

\section{Summary of tableau types included}
\begin{frame}{Summary: tableaux included in this file}
  \begin{enumerate}
    \item Initial augmented simplex tableau (compact).
    \item Given numeric optimal tableau (representative).
    \item Modified tableau with \(2000+\theta\) (objective-row shift).
    \item General pivot/record matrix expression showing \(c - A^T y\).
    \item Objective-row with \(\Delta c + (c - A^{T}y)\) explanation.
    \item Perturbed tableau for \(c_3 = 5000 + \theta\) and the pivot to eliminate \(\theta\).
    \item Resulting objective-row after elimination and the derived feasible interval for \(c_3\).
    \item Representative calculations and ranges for \(c_4\) and \(c_2\).
    \item Conceptual RHS perturbation tableau effect and marginal value formula.
  \end{enumerate}
  These cover all distinct tableau types and algebraic manipulations shown in the original PDF but omit exact duplicated frames.
\end{frame}

\begin{frame}{Practical takeaways}
  \begin{itemize}
    \item Reduced costs read directly from the final tableau give break-even adjustments for nonbasic variables.
    \item Adding \(\Delta c\) to \(c\) simply adds \(\Delta c\) to the objective row of the final tableau (then pivot if necessary).
    \item To analyze RHS changes use \(x_B = R b\) and dual vector \(y\); objective change \(\Delta z = y^\top \Delta b\) while the basis remains valid.
  \end{itemize}
\end{frame}

\begin{frame}{Questions}
  \centering Thank you — any further edits or additional tableaux you want inserted verbatim?
\end{frame}

\end{document}
