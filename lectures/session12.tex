
\section{Session 12. Sensitivity Analysis}
%%%%%%%%%%%%%%%%%%%%%%%%%%%%%%%%%%%%%%%%%%%%%%%%%%%%%%%%%%%%%%%%%%%%%%%%%%%%%
%%%%%%%%%%%%%%%%%%%%%%%%%%%%%%%%%%%%%%%%%%%%%%%%%%%%%%%%%%%%%%%%%%%%%%%%%%%%%
%%%%%%%%%%%%%%%%%%%%%%%%%%%%%%%%%%%%%%%%%%%%%%%%%%%%%%%%%%%%%%%%%%%%%%%%%%%%%

\begin{itemize}
  \item Understanding the concept of postoptimality analysis
  \item Understanding and solving LP problems requiring sensitvity analysis
  \item Understanding the matrix representation of the Simplex solution
\end{itemize}

\begin{itemize}
  \item After an optimal solution is found, the analyst needs to review the problem parameters and the solution.
  \item This process is called {\bf postoptimality analysis}:
  \begin{itemize}
    \item confirming or updating problem parameters (cost and availabilities of activities and resources)
    \item if changes need to be introduced in the original parameters, assessing their impact on the optimality of the solution.
  \end{itemize}
  \item If changes are small, re-optimization may not be needed.
  \item {\bf Sensitivity analysis} is the study of the effect that types, ranges, and magnitude of changes in problem parameters have in the value of the objective function, {\em without the need to solve again the new linear problem}.
\end{itemize}

  In a LP problem, 
\begin{eqnarray*}
 \text{max }\quad\uvec{c}^t \uvec{x}\\
 \text{subject to }\quad A\uvec{x}\leq \uvec{b}, \forall x_i\geq0
\end{eqnarray*}

one can have two situations: one may have interest in knowing what happens with modifications in the parameters $c$ or modifications in the parameters $b$.

In addition, one can explore what occurs when adding an extra constraint or adding a new variable.



  \begin{Exercise}
    A farmer wants to minimize the cost of the food given to her lifestock. Two different types of nutrients $A$ and $B$ are needed by the animals, and she needs a minimum nutrition to be achieved.
    \begin{center}
    \begin{tabular}{c|c|c|c}
     & A & B & price\\
     \hline
     Feed 1 & 10 & 3 & 16\\
     Feed 2 & 4 & 5 & 14\\
     \hline
     requirements & 124 & 60 & \\
    \end{tabular}
    \end{center}

    Find how resilient is she to the changes in the price? What happens with respect to basic vs non-basic variables in a general case?

  \end{Exercise}
\footnote{veure  \url{https://youtu.be/o1pznRt_-y0?t=3378} minut 50 i següent capítol}


  In such cases we will explore the coeficients of the objective function in the dual problem, instead.

  \begin{Exercise}
    Given the LP problem
    \begin{equation*}
      \begin{aligned}
        \text{minimize } \quad & z = 16x_1+14x_2 \\
        \text{subject to }\quad &
        \begin{array}{rcl}
          10x_1 + 4x_2 &\geq &124 \\
          3x_1 + 5x_2 &\geq &60 \\
          x_1,x_2 &\geq& 0
        \end{array}
      \end{aligned}
    \end{equation*}

    Explore the sensitivity of the minimum value of $z$ with respect to the parameters in the RHS of the contraints. Hint: consider the dual problem.

  \end{Exercise}




\footnote{venim de \url{https://youtu.be/o1pznRt_-y0?t=1603} i \url{https://www.youtube.com/watch?v=yU8updOR87c}}


  Solving the dual problem gives a lot of insight into the actual situation.

  \begin{Exercise}*
  In a company that manufactures two types of bikes, $A$ and $B$, the owners want to maximize the benefits, taking into account that the production depends on a restricted amount of titanium, the time the machines can devote to the work and the manpower, which are all given below:
  \begin{center}
  \begin{tabular}{c|c|c|c}
   & A & B & limit\\
   \hline
   Titanium & 50 & 30 & 2000\\
    Machine time & 6 & 5 & 300\\
    Labor & 3 & 5 &200\\
   \hline
   price & 50 & 60 & \\
  \end{tabular}
  \end{center}

  %Check the link https://app.wooclap.com/PDHGSS and answer the questions


  \end{Exercise}



  Find \href{https://github.com/Biocomputing-Teaching/ORcourse/blob/main/code/linear_programming.ipynb}{\em here} a code with the solution using ORtools.
  \\[10pt]
  The Sensitivity analysis tool in Excel is nicely explained \href{https://www.youtube.com/watch?v=zKqU5NGE-t0&list=PLjiMsqjDUvBiArYMqCGZSDJNgEMPALdQu&index=1}{\em here}.
\\[10pt]
  Notice that in a \href{https://youtu.be/v2ZuKwAaXOI}{non-linear scenario}, what we have identified here as {\em reduced costs} are in fact {\em reduced gradient} (or actual gradient of the optimal solution) while {\em shadow prices} (margin for profit to be obtained by changing 1 unit in each constraint RHS term) correspond to the {\em Lagrange multipliers}.



%----------------------------------------------------------------------------------------





