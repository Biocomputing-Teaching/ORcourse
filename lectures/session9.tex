
\section{Initial solutions for General Constraints: Artificial Variables}

additional material: artif.pdf

  When adding slack variables, we end up finding an initial feasible set of basic variables. But what happens when not all constraints are of the form "$\leq$"?\cite{carter_operations_2019}

  \begin{enumerate}
    \item Make all right hand sides $b_i$ of the constraint (in)equalities positive. If they are not, multiply the expression by $(-1)$.
    \item If we end up just with slack variables, they can be directly used as initial basic variables. If the expressions are of the type $\geq$ the selection of initial basic variables is not trivial.
    \item We then create {\bf artificial variables}, as a trick to create a starting basic solution
  \end{enumerate}
  \begin{equation*}
    \begin{aligned}
      \text{maximize } \quad & z = x_1+3x_2 \\
      \text{subject to }\quad &
      \begin{array}{rcl}
        2x_1 -x_2 &\leq &-1 \\
        x_1+x_2 &= &3 \\
        x_1,x_2 &\geq& 0
      \end{array}
    \end{aligned}
  \end{equation*}

  By leaving the right hand side of the constraints positive and adding a surplus variable, we end up having
  \begin{equation*}
    \begin{aligned}
       \text{subject to }\quad &
      \begin{array}{rcl}
        -2x_1 +x_2 -s_1 &= &1 \\
        x_1+x_2 &= &3
      \end{array}
    \end{aligned}
  \end{equation*}
  and now we introduce the artificial variables to create an initial solution:
  \begin{equation*}
    \begin{aligned}
       \text{subject to }\quad &
      \begin{array}{rcl}
        -2x_1 +x_2 -s_1 +R_1&= &1 \\
        x_1+x_2 +R_2&= &3
      \end{array}
    \end{aligned}
  \end{equation*}
  with all variables non-negative. Recall that, in fact, $R_1=R_2=0$, so they will be just used temporarily.
  \\[10pt]

  There are two main methods to solve the problem: the two-phase method, explained next, and the \href{https://www.youtube.com/watch?v=btjxqq-vMOg}{big-M method}, that we are not explaining here.


  \begin{enumerate}
    \item First we will use the Simplex method to minimize the values of $R_1$ and $R_2$. 
    \item If they reach the value of zero, there is a solution for the original problem. If not, there is no feasible solution to the original problem.
  \end{enumerate}
  In the above example, we want to minimize the function $z_R=R_1+R_2$ or, what is the same, we want to maximize the function $z_R=-R_1-R_2$
  \begin{equation*}
    \begin{array}{c}
    \\
    z_R\\
    R_1\\
    R_2\\
    \end{array}
    \begin{array}{ccccc|c}
       x_1 & x_2 & s_1 & R_1 & R_2 & Solution \\ \hline
       0 & 0 & 0 & 1 & 1 & 0 \\ \hline
       -2 & 1 & -1 & 1 & 0 & 1  \\
       1 & 1 & 0 & 0 & 1 & 3 \\
    \end{array}
    \end{equation*}

  Next, we transform the problem in order to get  basic variables (columns of one 1 and the rest zeros). For example, by using the second and third row to make appear 0 in the first row for the artificial variables:
  \begin{equation*}
    \begin{array}{c}
    \\
    z_R\\
    R_1\\
    R_2\\
    \end{array}
    \begin{array}{ccccc|c}
       x_1 & x_2 & s_1 & R_1 & R_2 & Solution \\ \hline
       1 & -2 & 1 & 0 & 0 & -4 \\ \hline
       -2 & 1 & -1 & 1 & 0 & 1  \\
       1 & 1 & 0 & 0 & 1 & 3 \\
    \end{array}
  \end{equation*}
  We can solve this problem with the Simplex method, obtaining a tableau:
  \begin{equation*}
    \begin{array}{c}
    \\
    z_R\\
    x_2\\
    x_1\\
    \end{array}
    \begin{array}{ccccc|c}
       x_1 & x_2 & s_1 & R_1 & R_2 & Solution \\ \hline
       0 & 0 & 0 & 1 & 1 & 0 \\ \hline
       0 & 1 & -1/3 & 1/3 & 2/3 & 7/3  \\
       1 & 0 & 1/3 & -1/3 & 1/3 & 2/3 \\
    \end{array}
  \end{equation*}
    which provides the optimal solution for Phase 1. This means that the original problem has solution (as $R_1=R_2=0$). In Phase 2, we will replace the first row by the original maximization problem and leave out the artificial variables, solving the now standard Simplex problem:
    \begin{equation*}
      \begin{array}{c}
        \\
        z\\
        x_2\\
        x_1\\
      \end{array}
      \begin{array}{ccc|c}
        x_1 & x_2 & s_1 &  Solution \\ \hline
        -1 & -3 & 0 &  0 \\ \hline
        0 & 1 & -1/3 & 7/3  \\
        1 & 0 & 1/3  & 2/3 \\
      \end{array}
    \end{equation*}
    that leads to
    \begin{equation*}
      \begin{array}{c}
        \\
        z\\
        x_2\\
        s_1\\
      \end{array}
      \begin{array}{ccc|c}
        x_1 & x_2 & s_1 &  Solution \\ \hline
        2 & 0 & 0 &  9 \\ \hline
        1 & 1 & 0 & 3  \\
        3 & 0 & 1  & 2 \\
      \end{array}
    \end{equation*}

      Draw the graphical solution of this problem and interpret the results.
