\Exercise Maximize $f(x,y)=xy$ subject to $100 \geq x+y$ and $x\leq 40$

\Answer 

The Karush Kuhn Tucker (KKT) conditions  for optimality are a set of necessary conditions for a solution to be optimal in a mathematical optimization problem. They are necessary and sufficient conditions for a local minimum in nonlinear programming problems. The KKT conditions consist of the following elements:

For an optimization problem in its standard form:

\begin{equation*}
  \begin{aligned}
    \text{max} \, f(x) \\
    \text{s.t.}\quad &
    \begin{array}{rcl}
      g_i(x)-b_i  & \leq & 0 \quad i=1,\ldots,k \\
      g_i(x)-b_i  & = & 0 \quad i=k+1,\ldots,m \\
    \end{array}
  \end{aligned}
\end{equation*}

There are 4 KKT conditions for optimal primal ($x4$) and dual ($\lambda$) variables. If $x^*$ denotes optimal values:
\begin{enumerate}
  \item Primal feasibility: all constraints must be satisfied: $g_i(x^*)-b_i$ is feasible. Applies to both equality and non-equality constraints.
  \item Gradient condition or No feasible descent: No possible improvement at the solution: 
  \[ \nabla f(x^*)-\sum_{i=1}^m \lambda_i^* \nabla g_i (x^*)=0\]
  \item Complementariety slackness: 
  \[\lambda_i^* (g_i(x^*)-b_i)=0\]
  \item Dual feasibility: $\lambda_i^*\geq 0$
\end{enumerate}

The last two conditions (3 and 4) are only required with inequality constraints and enforce a positive Lagrange multiplier when the constraint is active (=0) and a zero Lagrange multiplier when the constraint is inactive (>0). 

To solve our problem, first we will put it in its standard form:


\begin{equation*}
  \begin{aligned}
    \text{max} \, f(x,y)=xy \\
    \text{s.t.}\quad &
    \begin{array}{rcl}
      g_1(x,y)=x+y-100  & \leq & 0  \\
      g_2(x,y)=x-40 & \leq & 0  \\
      g_3(x,y)=-x&\leq&0\\
      g_4(x,y)=-y&\leq&0\\
    \end{array}
  \end{aligned}
\end{equation*}

\begin{center}
\begin{tikzpicture}
  \begin{axis}
    [ xmin=-30,xmax=120,
      ymin=-30,ymax=120,
      grid=both,
      grid style={line width=.1pt, draw=darkgray!10},
      major grid style={line width=.2pt,draw=darkgray!50},
      axis lines=middle,
      minor tick num=4,
      enlargelimits={abs=0.5},
      axis line style={latex-latex},
      samples=100,
      domain = -20:110,
      view={0}{90}
    ]
    \fill[blue, pattern=north west lines, pattern color=blue] (0,0) -- (0,100) -- (40,60) -- (40,0) -- (0,0);
    \draw[black] (40,-20) -- (40,80);
    \draw[black] (-20,120) -- (120,-20);
    \addplot3 [contour lua={number=15}
              ] {x*y};
  \end{axis}
\end{tikzpicture}
\end{center}

We will go through the different conditions:

 \begin{itemize}
  \item on the gradient:
  \[
    \begin{pmatrix}\pdv{f}{x}\\\pdv{f}{y}\end{pmatrix} 
    -\lambda_1 \begin{pmatrix}\pdv{g_1}{x}\\\pdv{g_1}{y}\end{pmatrix}
    -\lambda_2 \begin{pmatrix}\pdv{g_2}{x}\\\pdv{g_2}{y}\end{pmatrix} 
    -\lambda_3 \begin{pmatrix}\pdv{g_3}{x}\\\pdv{g_3}{y}\end{pmatrix}
    -\lambda_4 \begin{pmatrix}\pdv{g_4}{x}\\\pdv{g_4}{y}\end{pmatrix}=0
  \]
  which, in this example, resolves into:
  \begin{equation}\label{Eq:KKT1}y-(\lambda_1+\lambda_2-\lambda_3)=0\end{equation}
  \begin{equation}\label{Eq:KKT2}x-(\lambda_1-\lambda_4)=0\end{equation}
  \item on the complementary slackness:
  \begin{equation}\label{Eq:KKT3}\lambda_1 (x +y -100)=0\end{equation}
  \begin{equation}\label{Eq:KKT4}\lambda_2 (x-40)=0\end{equation}
  \begin{equation}\label{Eq:KKT5}\lambda_3 x =0\end{equation}
  \begin{equation}\label{Eq:KKT6}\lambda_4 y =0\end{equation}
  \item on the constraints:
  \begin{equation}\label{Eq:KKT7}x+ y \leq100\end{equation}
  \begin{equation}\label{Eq:KKT8}x\leq40\end{equation}
  \begin{equation}\label{Eq:KKT9}-x\leq0\end{equation}
  \begin{equation}\label{Eq:KKT10}-y\leq0\end{equation}

  plus $\lambda_1,\lambda_2,\lambda_3,\lambda_4\geq 0$.

  We will start by checking Eq. \ref{Eq:KKT3}:
  \begin{itemize}
    \item  Let us seeg what occurs if $\lambda_1=0$. Then, from Eq. \ref{Eq:KKT2}, $x+\lambda_4=0$ which implies that $x=\lambda_4=0$\footnote{Recall that both variables and multiplieras must be positive or zero, so, the only possibility for the equation to fullfill is that both are zero.}. But, then, from Eq. \ref{Eq:KKT4} we obtain that $\lambda_2=0$ which, using Eq. \ref{Eq:KKT1} gives $y+\lambda_3=0\Rightarrow y=\lambda_3=0$. Indeed, the KKT conditions are satisfied when all variables and multipliers are zero, but it is not a maximum of the function (see figure above).
    \item So, let us see what happens if $x+y-100=0$ and consider the two possibilities for $x$:
    \begin{description}
      \item[Case $x=0$:] Then, $y=100$, which would lead (Eq. \ref{Eq:KKT6}) to $\lambda_4=0$ and (Eq. \ref{Eq:KKT2}) to $x=\lambda_1=0$, that was discussed in the previous item. So, we need top explore the other possibility for $x$.
      \item[Case $x>0$:] From Eq. \ref{Eq:KKT5} $\lambda_3=0$ and, from Eqs. \ref{Eq:KKT1} and \ref{Eq:KKT2}:
      \[\systeme{y=\lambda_1+\lambda_2,x=\lambda_1+\lambda_4}
      \] 
      let us try what happens if, e.g., $\lambda_2\neq 0$ (or, said in other words, if constraint \ref{Eq:KKT8} is active): $x=40$. As we know we do not want $\lambda_1=0$, from Eq. \ref{Eq:KKT3} we obtain $x+y-100=0\Rightarrow y=60$.
    \end{description}

    The point $(x,y)=(40,60)$ fullfills the KKT conditions and is a maximum in the constrained maximization problem.

 \end{itemize}
\end{itemize}