\Exercise Maximize $f(x,y)=xy$ subject to $100 \geq x+y$ and $x\leq 40$

\Answer 

The Karush Kuhn Tucker (KKT) conditions  for optimality are a set of necessary conditions for a solution to be optimal in a mathematical optimization problem. They are necessary and sufficient conditions for a local minimum in nonlinear programming problems. The KKT conditions consist of the following elements:

For an optimization problem in its standard form:

\begin{equation*}
  \begin{aligned}
    \text{min} f(x) \\
    \text{subject to }\quad &
    \begin{array}{rcl}
      g_i(x)-b_i  & \geq & 0 \quad i=1,\ldots,k \\
      g_i(x)-b_i  & = & 0 \quad i=k+1,\ldots,m \\
    \end{array}
  \end{aligned}
\end{equation*}

There are 4 KKT conditions for optimal primal ($x4$) and dual ($\lambda$) variables. If $x^*$ denotes optimal values:
\begin{enumerate}
  \item Primal feasibility: all constraints must be satisfied: $g_i(x^*)-b_i$ is feasible. Applies to both equality and non-equality constraints.
  \item Gradient condition or No feasible descent: No possible improvement at the solution: 
  \[ \nabla f(x^*)-\sum_{i=1}^m \lambda_i^* \nabla g_i (x^*)=0\]
  \item Complementariety slackness: 
  \[\lambda_i^* (g_i(x^*)-b_i)=0\]
  \item Dual feasibility: $\lambda_i^*\geq 0$
\end{enumerate}

The last two conditions (3 and 4) are only required with inequality constraints and enforce a positive Lagrange multiplier when the constraint is active (=0) and a zero Lagrange multiplier when the constraint is inactive (>0). 

to solve our problem, first we will put it in its standard form:


\begin{equation*}
  \begin{aligned}
    \text{min } f(x,y)=-xy \\
    \text{subject to }\quad &
    \begin{array}{rcl}
      -x-y+100  & \geq & 0  \\
      -x-40 & \geq & 0  \\
    \end{array}
  \end{aligned}
\end{equation*}

\begin{center}
\begin{tikzpicture}
  \begin{axis}
    [ xmin=-20,xmax=110,
      ymin=-20,ymax=110,
      grid=both,
      grid style={line width=.1pt, draw=darkgray!10},
      major grid style={line width=.2pt,draw=darkgray!50},
      axis lines=middle,
      minor tick num=4,
      enlargelimits={abs=0.5},
      axis line style={latex-latex},
      samples=100,
      domain = -20:110,
      view={0}{90}
    ]
    \fill[blue, pattern=north west lines, pattern color=blue] (0,0) -- (0,100) -- (40,60) -- (40,0) -- (0,0);
    \draw[black] (40,-10) -- (40,70);
    \draw[black] (-10,110) -- (110,-10);
    \addplot3 [contour lua={number=14}
              ] {-x*y};
  \end{axis}
\end{tikzpicture}
\end{center}

We will go through the different conditions:

\begin{enumerate}
  \item Primal feasibility:  $g_i(x^*)-b_i$ is feasible. 
  \[-x^* -y^* +100 =0\]
  \[-x^*-40=0\]
  \item Gradient condition or No feasible descent:  
  \[ \begin{pmatrix}\pdv{f}{x}\\\pdv{f}{y}\end{pmatrix} 
  -\lambda_1 \begin{pmatrix}\pdv{g_1}{x}\\\pdv{g_1}{y}\end{pmatrix}
  -\lambda_2 \begin{pmatrix}\pdv{g_2}{x}\\\pdv{g_2}{y}\end{pmatrix} =0\]
  which, in this example, resolves into:
  \[\systeme[xy\lambda_1\lambda_2]{-y+\lambda_1+\lambda_2=0,-x-\lambda_1=0 }\]
  \item Complementariety slackness: 
  \[\lambda_1^* (-x^* -y^* +100)=0\]
  \[\lambda_2^* (-x^*-40)=0\]
  \item Dual feasibility: $\lambda_1,\lambda_2\geq 0$
\end{enumerate}

We can put the resulting 5 expressions for conditions 1 and 2 into matrix form:

\[
  \begin{pmatrix} -1 & -1 & 0&0\\ -1&0&0&0\\ 0&-1&1&1\\ -1&0&-1&0 \end{pmatrix}
  \begin{pmatrix} x\\y\\\lambda_1\\\lambda_2\end{pmatrix}=
  \begin{pmatrix} -100\\ 40\\0\\0 \end{pmatrix}
\]

  % \begin{minipage}[t]{\linewidth}
  %   \vspace{-2ex}
  %   %\includegraphics[width=\textwidth]{../figures/Afiquadrilater.pdf}
  %   \captionof{figure}{Transformacions afins d'un quadrilàter de vèrtex $(4,0)$, $(2,2)$, $(1,1)$ i $(2,-1)$.}
  %   \label{fig:trafquadril}
  % \end{minipage}
