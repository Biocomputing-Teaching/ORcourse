\Exercise The figure shows a network on four nodes, including net demands on the vertex, $b_k$, and cost an capacity on the edges, $(c_{i,j},u_{i,j})$ . 


\begin{tikzpicture}[
  node distance =24mm and 12mm,
  C/.style = {circle, draw, minimum size=1em}
                      ] 
  \node (v1) [C, label=10] {1};
  \node (v2) [C, label=12, above right=of v1] {2};
  \node (v3) [C, label=14, below right=of v2] {3};
  \node (v4) [C, label=13, below right=of v1] {4};
  %
  \draw (v1)--node[left,above, sloped]{C$_4$}(v2);    
  \draw [thick,->,>=stealth, line width=1.5pt] (v2)  to[C$_4$] (v3);    
  \draw [thick,->,>=stealth, line width=1.5pt] (v1)  to["4"] (v4);     
  \draw [thick,->,>=stealth, line width=1.5pt] (v4)  to["55"] (v3);   
      \end{tikzpicture}

\Answer 


The Karush Kuhn Tucker (KKT) conditions  for optimality are a set of necessary conditions for a solution to be optimal in a mathematical optimization problem. They are necessary and sufficient conditions for a local minimum in nonlinear programming problems. The KKT conditions consist of the following elements:

For an optimization problem in its standard form:

\begin{equation*}
  \begin{aligned}
    \text{min} f(x) \\
    \text{subject to }\quad &
    \begin{array}{rcl}
      g_i(x)-b_i  & \geq & 0 \quad i=1,\ldots,k \\
      g_i(x)-b_i  & = & 0 \quad i=k+1,\ldots,m \\
    \end{array}
  \end{aligned}
\end{equation*}

There are 4 KKT conditions for optimal primal ($x4$) and dual ($\lambda$) variables. If $x^*$ denotes optimal values:
\begin{enumerate}
  \item Primal feasibility: all constraints must be satisfied: $g_i(x^*)-b_i$ is feasible. Applies to both equality and non-equality constraints.
  \item Gradient condition or No feasible descent: No possible improvement at the solution: 
  \[ \nabla f(x^*)-\sum_{i=1}^m \lambda_i^* \nabla g_i (x^*)=0\]
  \item Complementariety slackness: 
  \[\lambda_i^* (g_i(x^*)-b_i)=0\]
  \item Dual feasibility: $\lambda_i^*\geq 0$
\end{enumerate}

The last two conditions (3 and 4) are only required with inequality constraints and enforce a positive Lagrange multiplier when the constraint is active (=0) and a zero Lagrange multiplier when the constraint is inactive (>0). 

to solve our problem, first we will put it in its standard form:


\begin{equation*}
  \begin{aligned}
    \text{min } f(x,y)=-xy \\
    \text{subject to }\quad &
    \begin{array}{rcl}
      -x-y+100  & \geq & 0  \\
      -x-40 & \geq & 0  \\
    \end{array}
  \end{aligned}
\end{equation*}

We will go through the different conditions:

\begin{enumerate}
  \item Primal feasibility:  $g_i(x^*)-b_i$ is feasible. 
  \[-x^* -y^* +100 =0\]
  \[-x^*-40=0\]
  \item Gradient condition or No feasible descent:  
  \[ \begin{pmatrix}\pdv{f}{x}\\\pdv{f}{y}\end{pmatrix} 
  -\lambda_1 \begin{pmatrix}\pdv{g_1}{x}\\\pdv{g_1}{y}\end{pmatrix}
  -\lambda_2 \begin{pmatrix}\pdv{g_2}{x}\\\pdv{g_2}{y}\end{pmatrix} =0\]
  which, in this example, resolves into:
  \[\systeme[xy\lambda_1\lambda_2]{-y+\lambda_1+\lambda_2=0,-x-\lambda_1=0 }\]
  \item Complementariety slackness: 
  \[\lambda_1^* (-x^* -y^* +100)=0\]
  \[\lambda_2^* (-x^*-40)=0\]
  \item Dual feasibility: $\lambda_1,\lambda_2\geq 0$
\end{enumerate}

We can put the resulting 5 expressions for conditions 1 and 2 into matrix form:

\[
  \begin{pmatrix} -1 & -1 & 0&0\\ -1&0&0&0\\ 0&-1&1&1\\ -1&0&-1&0 \end{pmatrix}
  \begin{pmatrix} x\\y\\\lambda_1\\\lambda_2\end{pmatrix}=
  \begin{pmatrix} -100\\ 40\\0\\0 \end{pmatrix}
\]

  % \begin{minipage}[t]{\linewidth}
  %   \vspace{-2ex}
  %   %\includegraphics[width=\textwidth]{../figures/Afiquadrilater.pdf}
  %   \captionof{figure}{Transformacions afins d'un quadrilàter de vèrtex $(4,0)$, $(2,2)$, $(1,1)$ i $(2,-1)$.}
  %   \label{fig:trafquadril}
  % \end{minipage}
